\documentclass[letterpaper]{Formatting} 
\usepackage{textcase}
\usepackage[usenames,dvipsnames]{xcolor}
\usepackage{hyperref}
\usepackage{multicol}
\definecolor{linkcolour}{rgb}{0.25,0.25,0.6}
\hypersetup{colorlinks,breaklinks,urlcolor=linkcolour,linkcolor=linkcolour}
\begin{document}

%----------------------------------------------------------------------------------------
%	TITLE SECTION
%----------------------------------------------------------------------------------------


\namesection{Rohan}{Bavishi}{ 
\href{mailto:rbavishi@iitk.ac.in}{rbavishi@iitk.ac.in} | \color{subheadings}\small+91-77-528-46562 
}

%----------------------------------------------------------------------------------------
%	LEFT COLUMN
%----------------------------------------------------------------------------------------
\begin{minipage}[t]{0.3\textwidth}% The left column takes up 30% of the text width of the page

%------------------------------------------------
% Contact & Profiles
%------------------------------------------------

\section{Contact \& Profiles} 
\runsubsection{Address}\\
A-302, Hall-2, IIT Kanpur\\Kanpur, Uttar Pradesh\\\\
\runsubsection{Alternate Email}
\href{mailto:rohan.bavishi95@gmail.com}{rohan.bavishi95@gmail.com}\\\\
%\runsubsection{L\MakeTextLowercase{inked}i\MakeTextLowercase{n}}
%\includegraphics[scale=0.4]{linkedin.jpg}
%\location{\includegraphics[scale=0.06]{linkedin.png} \hspace{0.7ex}| \href{http://in.linkedin.com/in/rbavishi}{rbavishi}}\vspace{0.5ex}
%\runsubsection{G\MakeTextLowercase{ithub}}
%\includegraphics[scale=0.05]{github.png} 
%\location{|    \href{https://github.com/rbavishi/}{rbavishi}}
\location{\includegraphics[scale=0.06]{linkedin.png} \hspace{0.7ex}| \href{http://in.linkedin.com/in/rbavishi}{rbavishi} \hspace{3ex} \includegraphics[scale=0.05]{github.png} | \href{https://github.com/rbavishi/}{rbavishi}}

\sectionspace % Some whitespace after the section
\vspace{-2ex}
\rule{5cm}{0.5pt}
\vspace{2ex}

%------------------------------------------------
% Coursework
%------------------------------------------------

\section{Coursework (Grade)}
Fundamentals of Computing (A*)\\
Data Structures \& Algorithms (A*)\\
Discrete Mathematics (A)\\
Introduction to Electronics (A)\\
Mathematics - Calculus (A)\\
Mathematics - Linear Algebra (A)\\
Logic for Computer Science (Ongoing)\\
Abstract Algebra (Ongoing)\\
Computer Organisation (Ongoing)\\\vspace{1ex}
\runsubsection{\footnotesize Complete Transcript | } \href{https://drive.google.com/file/d/0B0--s-r8CTxgRHdGMm1iRnQxelU/view}{Link}
\sectionspace % Some whitespace after the section
\vspace{-1ex}
\rule{5cm}{0.5pt}
\vspace{2ex}
%------------------------------------------------
% Skills
%------------------------------------------------

\section{Skills}


\location{Proficient:}
C \textbullet{} C++ \textbullet{} Python \textbullet{} \LaTeX \textbullet{} Bash \textbullet{} HTML\\
Git \textbullet{} Icarus Verilog \\
\location{Familiar:}
Java \textbullet{} JavaScript \textbullet{} Android\\
\location{Softwares:}
Photoshop \textbullet{} MATLAB
\sectionspace % Some whitespace after the section
\vspace{-1.5ex}
\rule{5cm}{0.5pt}
\vspace{2ex}

%------------------------------------------------
%Programming Activities
%------------------------------------------------

\section{Programming Activities}
\runsubsection{SPOJ}\\\vspace{0.5ex}
\small\uppercase {Solved: }\href{http://www.spoj.com/users/rbavishi/}{144} | \uppercase{World Rank: 870}\\
\vspace{2ex}
\runsubsection{Project Euler}\\\vspace{0.5ex}
\small \uppercase{Solved: }\href{https://projecteuler.net/profile/RJBavishi.png}{228/502} \\
\small \uppercase{India Rank : 16 | World : 815}
\vspace{1ex}
\rule{5cm}{0.5pt}
\vspace{2ex}

%-------------------------------------------------
%POR
%-------------------------------------------------

\section{Positions of Responsibility}
\runsubsection{Academic Mentor}\\
\descript{Counselling Service}\vspace{1ex}
\footnotesize \textbullet{} Helping academically weaker students with coursework by organizing quizzes and doubt-sessions\\
\vspace{3ex}
\runsubsection{Secretary}\\
\descript{Quiz Club,  IIT Kanpur}\vspace{1ex}
\footnotesize \textbullet{} Organising and participating in quizzes of various genre\\
\vspace{1ex}
\rule{5cm}{0.5pt}
%----------------------------------------------------------------------------------------
\end{minipage} % The end of the left column
\hfill
\vrule
\hspace{3ex}
%
%----------------------------------------------------------------------------------------
%	RIGHT COLUMN
%----------------------------------------------------------------------------------------
%
\begin{minipage}[t]{0.66\textwidth} % The right column takes up 66% of the text width of the page

%------------------------------------------------
% Education
%------------------------------------------------

\section{Education}

\runsubsection{B.Tech | Computer Science}
\descript{| Indian Insitute of Technology, Kanpur}

\location{Expected July 2013 – Aug 2017 | GPA : 10.0/10.0 (Overall)}
%\vspace{\topsep} % Hacky fix for awkward extra vertical space
\sectionspace % Some whitespace after the section

%------------------------------------------------

\runsubsection{HSC | Class 12}
\descript{| Shivaji Science College, Nagpur}
\location{May 2013 |  Aggregate : 90.16\%}
\sectionspace % Some whitespace after the section

%------------------------------------------------

\runsubsection{AISSCE | Class 10}
\descript{| Modern School, Nagpur}
\location{May 2011 | Aggregate : 96.4\%}
\sectionspace % Some whitespace after the section
\rule{12cm}{0.5pt}\vspace{3ex}
%------------------------------------------------
% Projects
%------------------------------------------------

\section{Academic Projects}

\runsubsection{Median Algorithms for Disk-Resident Data}\\

\location{Aug 2014 – Nov 2014 | Under \href{http://www.cse.iitk.ac.in/users/sbaswana/}{Prof. Surender Baswana} | Report Link : \href{https://drive.google.com/file/d/0B0--s-r8CTxgZTU3RzB3YnhMMVU/view}{Project Report} Repository : \href{https://github.com/rbavishi/Median-Algorithms-Disk-Resident-Data}{Github Link}}\vspace{1ex}
Independently developed a 2-pass deterministic and a 2-pass randomized algorithm to find median of large data-sets (1 Terabyte) with performance tests\\\vspace{1ex}
Deterministic Algorithm :\vspace{1ex}
\vspace{\topsep} % Hacky fix for awkward extra vertical space
\footnotesize {
\begin{tightitemize}
\item Similar to the original paper by Munro-Paterson(1980)(\textbf{\href{http://polylogblog.files.wordpress.com/2009/08/80munro-median.pdf}{Link}})
\item An $\epsilon$-approximate median ($\epsilon$ = 1/ n) calculated in the first pass, followed by the computation of the exact median in the second
\end{tightitemize}
}
\vspace{2ex}
\small Randomized Algorithm :\vspace{0.5ex}
\vspace{\topsep} % Hacky fix for awkward extra vertical space
\footnotesize{
\begin{tightitemize}
\item Basic Random-Sampling techniques using Mersenne-Twister PRNG implemented. Theoretical success probability calculated and compared with the practical performance over thousands of program-runs
\item Success probability close to 0.6 achieved for medium-sized data-sets (10-50 GB)
\end{tightitemize}
}
\sectionspace % Some whitespace after the section
\vspace{2ex}
%------------------------------------------------

\runsubsection{Peer-to-Peer Dropbox}\\
\location{Aug 2013 – Nov 2013 | Under \href{http://www.cse.iitk.ac.in/users/subhajit/}{Prof. Subhajit Roy} | Github Link : \href{https://github.com/rbavishi/P2P-Dropbox}{P2P Dropbox}}\vspace{1ex}
A Linux application for back-up and syncing of files between two or more peers
\vspace{\topsep} % Hacky fix for awkward extra vertical space
\footnotesize{
\begin{tightitemize}
\item Users have a shared folder across different machines, with local copies, in which any changes made are synced across all devices
\item Linux \href{http://man7.org/linux/man-pages/man7/inotify.7.html}{\textbf{inotify}} API used to track changes in the shared folder
\item \href{http://linux.about.com/library/cmd/blcmdl1_rsync.htm}{rsync} used to sync changes in files/folders to ensure efficient transfer
\item Multithreading with mutexes used to parallelize syncing operations
\item Retry mechanisms and network detection system to ensure syncing even in networks with inter-mittent connectivity
\item Command-Line-Interface to add/delete folders and show synchronization status
\end{tightitemize}
}
\sectionspace % Some whitespace after the section
\vspace{2ex}
\runsubsection{Frama-C Program Verification}\\
\location{Jan 2014 - March 2014 | Under \href{http://www.cse.iitk.ac.in/users/subhajit/}{Prof. Subhajit Roy}}\vspace{1ex}
A reading/implementation project to verify program modules using \href{http://frama-c.com/}{Frama-C}
\vspace{\topsep}
\footnotesize{
\begin{tightitemize}
\item Verified several common programs using \href{frama-c.com/acsl.html}{ACSL} language specification such as sorting/searching, median-finding, k th order statistic etc
\item Proved formal properties of common implementations like bubble-sort using the \href{krakatoa.lri.fr/jessie.pdf}{Jessie} plugin for deductive verification
\end{tightitemize}
}
\sectionspace
\vspace{2ex}
\rule{12cm}{0.5pt}

%------------------------------------------------
% Awards
%------------------------------------------------
\vspace{3.9ex}
\section{Awards \& Achievements} 
\vspace{\topsep}
\small{
\begin{tightitemize}
\item Secured an All-India-Rank of 202 in JEE Advanced 2013 amongst 150,000 candidates\\
\item Secured an All-India-Rank of 175 in JEE Mains 2013 amongst 20,00,000 candidates\\
\item Secured an All-India-Rank of 33 in AMTI - Mathematics Olympiad 2013\\
\item Best Overall Student (2011-2013) - Shivaji Science College, Nagpur \\
\end{tightitemize}
}
\sectionspace % Some whitespace after the section
\vspace{2ex}
\rule{12cm}{0.25pt}
\end{minipage} % The end of the right column
%----------------------------------------------------------------------------------------
%	SECOND PAGE (If need be)
%----------------------------------------------------------------------------------------

%\newpage 

%\begin{minipage}[t]{0.30\textwidth} % The left column takes up 30% of the text width of the page.

%\section{}

%\end{minipage} % The end of the left column
%\hfill
%\begin{minipage}[t]{0.66\textwidth} % The right column takes up 66% of the text width of the page

%\section{}

%\end{minipage} % The end of the right column

%----------------------------------------------------------------------------------------

\end{document}
