%%%%%%%%%%%%%%%%%%%%%%%%%%%%%%%%%%%%%%%%%%%%%%%%%%%%%%%%%%%%%%%%%%%%%%%%%%%%%%%%
% Medium Length Graduate Curriculum Vitae
% LaTeX Template
% Version 1.2 (3/28/15)
%
% This template has been downloaded from:
% http://www.LaTeXTemplates.com
%
% Original author:
% Rensselaer Polytechnic Institute 
% (http://www.rpi.edu/dept/arc/training/latex/resumes/)
%
% Modified by:
% Daniel L Marks <xleafr@gmail.com> 3/28/2015
%
% Important note:
% This template requires the res.cls file to be in the same directory as the
% .tex file. The res.cls file provides the resume style used for structuring the
% document.
%
%%%%%%%%%%%%%%%%%%%%%%%%%%%%%%%%%%%%%%%%%%%%%%%%%%%%%%%%%%%%%%%%%%%%%%%%%%%%%%%%

%-------------------------------------------------------------------------------
%	PACKAGES AND OTHER DOCUMENT CONFIGURATIONS
%-------------------------------------------------------------------------------

%%%%%%%%%%%%%%%%%%%%%%%%%%%%%%%%%%%%%%%%%%%%%%%%%%%%%%%%%%%%%%%%%%%%%%%%%%%%%%%%
% You can have multiple style options the legal options ones are:
%
%   centered:	the name and address are centered at the top of the page 
%				(default)
%
%   line:		the name is the left with a horizontal line then the address to
%				the right
%
%   overlapped:	the section titles overlap the body text (default)
%
%   margin:		the section titles are to the left of the body text
%		
%   11pt:		use 11 point fonts instead of 10 point fonts
%
%   12pt:		use 12 point fonts instead of 10 point fonts
%
%%%%%%%%%%%%%%%%%%%%%%%%%%%%%%%%%%%%%%%%%%%%%%%%%%%%%%%%%%%%%%%%%%%%%%%%%%%%%%%%

\documentclass[mm]{simple_style}  

% Default font is the helvetica postscript font
\usepackage{helvet}
\usepackage{hyperref}
\usepackage{url}
\usepackage{xcolor}
\hypersetup {
    colorlinks=true,
    linkcolor=colorlink,
    filecolor=magenta,      
    urlcolor=colorlink,
}
\usepackage[left=0.7in, right=2in, top=1in]{geometry}

% Increase text height
\textheight=700pt

\begin{document}

%-------------------------------------------------------------------------------
%	NAME AND ADDRESS SECTION
%-------------------------------------------------------------------------------
\name{Rohan Bavishi}
\qualification{Senior Undergraduate, Computer Science, IIT Kanpur}
\emailone{rbavishi@iitk.ac.in}
\emailtwo{rohan.bavishi95@gmail.com}
\website{http://home.iitk.ac.in/~rbavishi}{\url{home.iitk.ac.in/~rbavishi}}
\phone{+91-73-180-18920}

% Note that addresses can be used for other contact information:
% -phone numbers
% -email addresses
% -linked-in profile

\address{C-324, Hall-1\\IIT Kanpur\\Kanpur, Uttar Pradesh, India}

% Uncomment to add a third address
%\address{Address 3 line 1\\Address 3 line 2\\Address 3 line 3}
%-------------------------------------------------------------------------------

\begin{resume}

%-------------------------------------------------------------------------------
%	EDUCATION SECTION
%-------------------------------------------------------------------------------
\section{Education}
\cusemph{Indian Institute of Technology Kanpur}, Uttar Pradesh, India\\
{\sl Bachelor of Technology}, Computer Science and Engineering, \timeline{Jul' 13 - Jul' 17}\\
\cusemph{GPA: 9.7/10} (Overall)\\
\sectionline
%-------------------------------------------------------------------------------

%-------------------------------------------------------------------------------
%	RESEARCH SECTION
%-------------------------------------------------------------------------------
\section{Research\\Interests}
\par
Program Analysis and Verification, Automated Debugging and Synthesis, \\Compiler Optimizations, Decision Procedures

\section{Publications}
\cusemph{Rohan Bavishi}, Awanish Pandey, Subhajit Roy, \textit{To Be Precise : Regression Aware Debugging} \\\cusemph{OOPSLA 2016} (To Appear), Amsterdam, Netherlands

\section{Research\\Projects}
\cusemph{New Strategy for analysis of Concurrent Programs via Sequentialization} \\
\supervisor{Supervisor : Prof. Subhajit Roy}\timeline{Aug '16 - Present}
\\\\
\cusemph{Using Interpolant-Based Proofs to Improve Bug Localization and Repair}
\\
\supervisor{Supervisor : Prof. Subhajit Roy}\timeline{Jul '15 - Aug '16}
\\\\
\cusemph{Using SAT/QBF-Solvers to Detect Side-Channel Vulnerabilities in Hardware}
\\
\supervisor{Supervisors : Prof. Paolo Ienne and Mr. Andrew Becker}\timeline{May '16 - Jul '16}
\\\\
\cusemph{A CBMC-based Implementation of DirectFix}
\\
\supervisor{Supervisor : Prof. Subhajit Roy}\timeline{May '15 - Jul '15}

\sectionline
%-------------------------------------------------------------------------------

%-------------------------------------------------------------------------------
%	ACADEMIC PROJECTS SECTION
%-------------------------------------------------------------------------------
\section{Academic\\Projects}
\cusemph{Re-Inventing A Median Algorithm for Disk-Resident Data}
\\
\supervisor{Supervisor : Prof. Surender Baswana}\timeline{Aug '14 - Nov '14}
\\\\
\cusemph{Peer-to-Peer Dropbox}
\\
\supervisor{Supervisor : Prof. Subhajit Roy}\timeline{Aug '13 - Nov '13}

\sectionline
%-------------------------------------------------------------------------------

%-------------------------------------------------------------------------------
%	COURSE PROJECTS SECTION
%-------------------------------------------------------------------------------
\section{Course\\Projects}
\cusemph{An End-to-End Compiler for Perl-like Language}
\\
\supervisor{Course : Compilers $|$ Supervisor : Prof. Subhajit Roy}\timeline{Jan '15 - Apr '15}
\\\\
\cusemph{An Integer Super-Scalar MIPS-R10K based Processor Simulator}
\\
\supervisor{Course : Computer Architecture $|$ Supervisor : Prof. Mainak Chaudhuri}\timeline{Jan '15 - Apr '15}

\sectionline
%-------------------------------------------------------------------------------

%-------------------------------------------------------------------------------
%	COMPUTER SKILLS SECTION
%-------------------------------------------------------------------------------
\section{Computer\\Skills}

\cusemph{Languages}: C, C++, Python, Assembly (x86, MIPS), Bash, Verilog, VHDL, \LaTeX.
\\
\cusemph{SAT/SMT Solvers}: MathSAT, Z3 
\\
\cusemph{Research Tools}: CBMC (Expert), KLEE, CSeq
\\
\sectionline
%-------------------------------------------------------------------------------

%-------------------------------------------------------------------------------
%	Interests
%-------------------------------------------------------------------------------
\section{Interests}
Interest One, Interest Two, Interest Three, Interest Four, ...Et Cetera.
%-------------------------------------------------------------------------------
\end{resume}
\end{document}
