%%%%%%%%%%%%%%%%%%%%%%%%%%%%%%%%%%%%%%%%%%%%%%%%%%%%%%%%%%%%%%%%%%%%%%%%%%%%%%%%
% Medium Length Graduate Curriculum Vitae
% LaTeX Template
% Version 1.2 (3/28/15)
%
% This template has been downloaded from:
% http://www.LaTeXTemplates.com
%
% Original author:
% Rensselaer Polytechnic Institute 
% (http://www.rpi.edu/dept/arc/training/latex/resumes/)
%
% Modified by:
% Daniel L Marks <xleafr@gmail.com> 3/28/2015
%
% Important note:
% This template requires the res.cls file to be in the same directory as the
% .tex file. The res.cls file provides the resume style used for structuring the
% document.
%
%%%%%%%%%%%%%%%%%%%%%%%%%%%%%%%%%%%%%%%%%%%%%%%%%%%%%%%%%%%%%%%%%%%%%%%%%%%%%%%%

%-------------------------------------------------------------------------------
%	PACKAGES AND OTHER DOCUMENT CONFIGURATIONS
%-------------------------------------------------------------------------------

%%%%%%%%%%%%%%%%%%%%%%%%%%%%%%%%%%%%%%%%%%%%%%%%%%%%%%%%%%%%%%%%%%%%%%%%%%%%%%%%
% You can have multiple style options the legal options ones are:
%
%   centered:	the name and address are centered at the top of the page 
%				(default)
%
%   line:		the name is the left with a horizontal line then the address to
%				the right
%
%   overlapped:	the section titles overlap the body text (default)
%
%   margin:		the section titles are to the left of the body text
%		
%   11pt:		use 11 point fonts instead of 10 point fonts
%
%   12pt:		use 12 point fonts instead of 10 point fonts
%
%%%%%%%%%%%%%%%%%%%%%%%%%%%%%%%%%%%%%%%%%%%%%%%%%%%%%%%%%%%%%%%%%%%%%%%%%%%%%%%%

\documentclass[mm]{simple_style}  

% Default font is the helvetica postscript font
\usepackage{helvet}
\usepackage{hyperref}
\usepackage{url}
\usepackage{xcolor}
\hypersetup {
    colorlinks=true,
    linkcolor=colorlink,
    filecolor=magenta,      
    urlcolor=colorlink,
}
\usepackage[left=0.7in, right=2in, top=0.9in]{geometry}

% Increase text height
\textheight=700pt

\begin{document}

%-------------------------------------------------------------------------------
%	NAME AND ADDRESS SECTION
%-------------------------------------------------------------------------------
\name{Rohan Bavishi}
\qualification{Senior Undergraduate, Computer Science, IIT Kanpur}
\emailone{rbavishi@iitk.ac.in}
\emailtwo{rohan.bavishi95@gmail.com}
\website{http://home.iitk.ac.in/~rbavishi}{\url{home.iitk.ac.in/~rbavishi}}
\phone{+91-73-180-18920}

% Note that addresses can be used for other contact information:
% -phone numbers
% -email addresses
% -linked-in profile

\address{C-324, Hall-1\\IIT Kanpur\\Kanpur, Uttar Pradesh, India}

% Uncomment to add a third address
%\address{Address 3 line 1\\Address 3 line 2\\Address 3 line 3}
%-------------------------------------------------------------------------------

\begin{resume}

%-------------------------------------------------------------------------------
%	EDUCATION SECTION
%-------------------------------------------------------------------------------
\section{Education}
\cusemph{Indian Institute of Technology Kanpur}, Uttar Pradesh, India\\
{\sl Bachelor of Technology}, Computer Science and Engineering, \timeline{Jul' 13 - Jul' 17}\\
\cusemph{GPA: 9.7/10} (Overall)\\
\sectionline
%-------------------------------------------------------------------------------

%-------------------------------------------------------------------------------
%	RESEARCH SECTION
%-------------------------------------------------------------------------------
\section{Research\\Interests}
\par
Program Analysis and Verification, Automated Debugging and Synthesis, \\Compiler Optimizations, Decision Procedures

\section{Publications}
\cusemph{Rohan Bavishi}, Awanish Pandey, Subhajit Roy, \textit{To Be Precise : Regression Aware Debugging} \\\cusemph{OOPSLA 2016} (To Appear), Amsterdam, Netherlands

\vspace{3ex}
\section{Research\\Projects}
\cusemph{New Strategy for analysis of Concurrent Programs via Sequentialization} \\
\supervisor{Supervisor : Prof. Subhajit Roy}\timeline{Aug '16 - Present}
\vspace{1ex}
\\
\describe{
	Working with \href{http://users.ecs.soton.ac.uk/gp4/cseq/cseq.html}{CSeq} for code-to-code translation of concurrent into equivalent sequential programs\\
	Devising solving strategies to reduce verification time on existing backends\\
}
\vspace{2ex}\\
\cusemph{Using Interpolant-Based Proofs to Improve Bug Localization and Repair (Publication)}
\\
\supervisor{Supervisor : Prof. Subhajit Roy}\timeline{Jul '15 - Aug '16}
\vspace{1ex}
\\
\describe{
	A new method to improve quality of bug localizations for a set of given passing and failing test-cases\\
	Interpolants are constructed from passing tests to derive \emph{soft} roadblocks which discourage localizations which violate these interpolants\\
	Improvements upto 45\% achieved in the quality of localizations in terms of developer effort involved\\
	\emph{Paper accepted in OOPSLA, one of the premier peer-reviewed conferences in Programming Languages}
}
\vspace{2ex}\\
\cusemph{Using SAT/QBF-Solvers to Detect Side-Channel Vulnerabilities in Hardware}
\\
\supervisor{Supervisors : Prof. Paolo Ienne and Mr. Andrew Becker}\timeline{May '16 - Jul '16}
\vspace{1ex}
\\
\describe{
	Summer Internship Project at Processor Architecture Laboratory, EPFL, Switzerland\\
	Studied various side-channel attacks, their mitigation techniques, and formal verification methods to prove their effectiveness\\
	Developed a QBF-Encoding to verify if a cryptographic circuit is secure against a popular side-channel attack based on fault-injection\\
	In the process of writing a paper and submitting to a peer-reviewed conference\\
}
\vspace{2ex}\\
\cusemph{A CBMC-based Implementation of DirectFix}
\\
\supervisor{Supervisor : Prof. Subhajit Roy}\timeline{May '15 - Jul '15}
\vspace{1ex}
\\
\describe{
	Ported the algorithm used in \href{https://www.comp.nus.edu.sg/~abhik/pdf/ICSE15-directfix.pdf}{DirectFix} to CBMC\\
	Tried to reproduce the experimental results provided in the paper, and devise further optimizations\\
}

\sectionline
%-------------------------------------------------------------------------------

%-------------------------------------------------------------------------------
%       AWARDS & ACHIEVEMENTS	
%-------------------------------------------------------------------------------
\section{Awards \& Achievements}
Secured an All-India-Rank of 202 in JEE Advanced 2013 amongst 150,000 candidates\\
Secured an All-India-Rank of 175 in JEE Mains 2013 amongst 20,00,000 candidates\\
Secured an All-India-Rank of 33 in AMTI-Mathematics Olympiad\\
Academic Excellence Award 2013-14, IIT Kanpur

\sectionline
%-------------------------------------------------------------------------------

\newpage
\sectionline

%-------------------------------------------------------------------------------
%	ACADEMIC PROJECTS SECTION
%-------------------------------------------------------------------------------
\section{Academic\\Projects}
\cusemph{Re-Inventing A Median Algorithm for Disk-Resident Data}
\\
\supervisor{Supervisor : Prof. Surender Baswana}\timeline{Aug '14 - Nov '14}
\vspace{1ex}
\\
\describe{
	Re-invented a two-pass \textit{deterministic} algorithm to find median of large data-sets (approx. 1 TB)\\
	The algorithm developed was similar to the original \href{http://polylogblog.files.wordpress.com/2009/08/80munro-median.pdf}{paper} by Munro-Paterson (1980)\\
	Carried out extensive tests to evaluate the performance of the algorithm
}\vspace{2ex}
\\
\cusemph{Peer-to-Peer Dropbox}
\\
\supervisor{Supervisor : Prof. Subhajit Roy}\timeline{Aug '13 - Nov '13}
\vspace{1ex}
\\
\describe{
	A linux application for backing-up and syncing of files between two or more peers\\
	Users have a shared folder across different machines, with local copies, in which any changes made are synced across all devices\\
	Linux inotify API used to track changes in the shared folder and rsync used to sync the modifications to ensure efficient transfer\\
	Multithreading with mutexes used to parallelize syncing and file-monitoring operations
}

\sectionline
%-------------------------------------------------------------------------------

%-------------------------------------------------------------------------------
%	COURSE PROJECTS SECTION
%-------------------------------------------------------------------------------
\section{Course\\Projects}
\cusemph{An End-to-End Compiler for Perl-like Language}
\\
\supervisor{Course : Compilers $|$ Supervisor : Prof. Subhajit Roy}\timeline{Jan '15 - Apr '15}
\vspace{1ex}
\\
\describe{
	Built an end-to-end compiler that takes a subset of the Perl language and outputs MIPS assembly\\
	Features such as \textit{operator overloading}, \textit{dynamic type-checking}, \textit{variable function arguments}, \textit{hashes}, \textit{lists}, \textit{type-based namespaces} implemented\\
	Secured maximum points possible for the project
}
\vspace{2ex}\\
\cusemph{An Integer Super-Scalar MIPS-R10K based Processor Simulator}
\\
\supervisor{Course : Computer Architecture $|$ Supervisor : Prof. Mainak Chaudhuri}\timeline{Jan '15 - Apr '15}

\sectionline
%-------------------------------------------------------------------------------

%-------------------------------------------------------------------------------
%	COMPUTER SKILLS SECTION
%-------------------------------------------------------------------------------
\section{Computer\\Skills}

\cusemph{Languages}: C, C++, Python, Assembly (x86, MIPS), Bash, Verilog, VHDL, \LaTeX.
\\
\cusemph{SAT/SMT Solvers}: MathSAT, Z3 
\\
\cusemph{Research Tools}: CBMC (Expert), KLEE, CSeq
\\
\sectionline
%-------------------------------------------------------------------------------

%-------------------------------------------------------------------------------
%	Interests
%-------------------------------------------------------------------------------
\section{Extra Interests}
\cusemph{Project Euler}: Solved : 257/Something \textit{(India Rank : 11)}\\
\cusemph{Programming Hobbies}: Competitive Programming, CTF, Open Source
%-------------------------------------------------------------------------------
\end{resume}
\end{document}
